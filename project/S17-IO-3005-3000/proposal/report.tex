\documentclass[9pt,twocolumn,twoside]{styles/osajnl}
\usepackage{fancyvrb}
\journal{i524} 

\title{Project Proposal for I524}

\author[1,*]{Abhishek Gupta}
\author[2, **]{Avadhoot Agasti}

\affil[1]{School of Informatics and Computing, Bloomington, IN 47408, U.S.A.}

\affil[*]{Corresponding authors: abhigupt@iu.edu}
\affil[**]{Corresponding authors: aagasti@iu.edu}

\dates{project-1: Data mining for a wiki url , \today}

\ociscodes{Cloud, I524}

% replace this with your url in github/gitlab
\doi{\url{https://github.com/cloudmesh/classes/blob/master/docs/source/format/report/report.pdf}}

\begin{abstract}
\end{abstract}

\setboolean{displaycopyright}{true}


\begin{document}

\maketitle

\section{Problem}

Given a wiki URL of a person, find out his details like School, Spouse, Coaches, language, alma-meter etc Typically, the wiki page has all this information available but in the free form text. We need to converting it into structured data format so that it can help us analyze the people, from the networks etc

\section{Solution}

Use tensor flow\cite{www-tensor-develop} to create word vectors. Train it using manual tagging and then use the model for analytics and prediction.

\begin{center}
 \begin{tabular}{||c c||} 
 \hline
 Technology Name & Purpose  \\ [0.5ex] 
 \hline\hline
 tensor flow & work vector and model \\ 
 \hline
 spark streaming \cite{www-tensor-spark} & realtime data analysis \\
 \hline
 ansible & automated deployment \\
 \hline
 python\cite{www-python-develop} & development \\
 \hline
 ansible\cite{www-python-develop} & deployment \\
 \hline
\end{tabular}
\end{center}

\section{Deployment}
Solution will deploy using Ansible playbook.

% Bibliography

\bibliography{references}
 
\paragraph{6. Conclusion}

Put in some conclusion based on what you have researched

\paragraph{Acknowledgement}

Put in the information for this class and who may sponsor
you. Examples will be given later

\end{document}
